% Wildlife Forensics Syllabus: Fall 2019

\documentclass[hidelinks]{article}

%-- Activate below to make default font sans serif --%
%\renewcommand{\familydefault}{\sfdefault}

%-- Set Margins --%
\usepackage[margin=2.5cm]{geometry}

%-- Package for text colouration --%
\usepackage[usenames,dvipsnames]{color} 

%-- Package for Links --%
\usepackage{hyperref} 

%-- Package for left-justifying columns of specified width --%
\usepackage{array}  % required for the raggedright function

%-- Set paragraph spacing and indentation --%
\setlength{\parindent}{0pt}
\setlength{\parskip}{0.5cm}

%-- Package for controlling spacing of lists --%
\usepackage[shortlabels]{enumitem}

%-- Packages for making nice tables --%
\usepackage{booktabs}
\usepackage{makecell}  % Required for specifying multiple rows within a table cell
\usepackage[table]{xcolor}  % Required for alternating line colours of tables

%-- Setting for headers ---%
\usepackage{fancyhdr}
\pagestyle{fancy}
\renewcommand{\headrulewidth}{0pt}  % Removes line under header
\lhead{\includegraphics[width=0.6\textwidth]{figures/logo.png}}

%-- Packages for Proper Handling of Figures --%
\usepackage{graphicx}
\usepackage{float}


%---------------------------------------------------------------------------------------
\begin{document}

%-- Without the *, vspace is cleared at the beginning of each page --%
\vspace*{0.01cm}

%-----------------------------------%
%  Course Info                         %
%-----------------------------------%
\begin{center}
	\Large{\textbf{Wildlife Forensics - FRSC/BIOL 4002\\
	Fall 2019\\
	Course Outline}}
\end{center}


%-----------------------------------%
%      Instructor Info              %
%-----------------------------------%
\textbf{\underline{Instructor Information}:}\\
\\
	%-- The @{} removes the spacing before that column, aligning these to the left (otherwise they are indented a bit) --%
	\begin{tabular}{@{} p{2.3cm} l }
		\textbf{Instructor:}	& Dr. Timothy R. Frasier (``Tim''), Associate Professor\\
		\\
		\textbf{Contact:} 	& Office: S327\\
					& E-mail: timothy.frasier@smu.ca\\
					& Tel: 902-491-6382\\
		\\
		\textbf{Office Hours:} & TR 1:00--4:00 or by appointment				
	\end{tabular}

	\begin{center}
		\rule{4cm}{0.5pt}
	\end{center}


%-----------------------------------%
%      Course Info                  %
%-----------------------------------%
\textbf{\underline{Course Information}:}


	%----------------------------------%
	%  Course times and locations      %
	%----------------------------------%
	\begin{tabular}{@{} p{2.3cm} l l l}
		\textbf{Lectures:} & MW & 1:00--2:15 & S345\\
		\textbf{Labs:} & W & 2:30--5:30 & S106\\
	\end{tabular}
	
	
	%----------------------------------%
	%  Course Sites                    %
	%----------------------------------%
	\begin{tabular}{@{} p{2.3cm} p{13.9cm}}
		\textbf{Course Site:} & This course has an associated Brightspace page. It will be the main way in which information and materials will be distributed to you: I will post readings, lecture slides, assignment instructions, and other course materials there. You are responsible for checking the site regularly.\\
	\end{tabular}


	%----------------------------------%
	%  Course Description              %
	%----------------------------------%
	\textbf{Course Description:}\\
	\begin{tabular}{@{} p{2.3cm} p{13.9cm}}
		 & The goals of this course are for students to learn about the techniques involved in wildlife forensics, how the resulting data are interpreted, and how this information is used in a legal setting. Although many aspects of wildlife forensics are covered, there is a focus on DNA methods.\\
	\end{tabular}

	%------------%
	% Credits    %
	%------------%
	\textbf{3 credit hours}

	%----------------------------------%
	%  Pre-requisites                  %
	%----------------------------------%
	\begin{tabular}{@{} p{2.6cm} l }
		\textbf{Pre-requisites:} & BIOL 2307 (Genetics)
	\end{tabular}

	%----------------------------------%
	%  Required Materials             %
	%----------------------------------%
	\begin{tabular}{@{} l l }
		\textbf{Required Materials:} & No textbook. Required readings will be provided as needed.
	\end{tabular}

	\newpage
	%------------------------------------%
	% Methods of Course Delivery %
	%------------------------------------%
	\textbf{Methods of Course Delivery:}\\
	Class periods will alternate between lectures and case studies. The lecture periods will focus on introducing major new ideas, concepts, and information that have general relevance to wildlife forensics. Every-other class period will focus on a particular case study, usually species-based, but sometimes concept-based. The first two of these will be led by the professor, and the rest will be led by students. For the laboratory component, students will learn all of the theory and techniques associated with molecular species identification, which will be applied to food purchased at local restaurants.\\
	
	
	%-----------------------------------%
	%      Learning Outcomes         %
	%-----------------------------------%
	\textbf{Learning Outcomes:}\\
	By the end of this course, students should be able to:
		\begin{itemize}[topsep=-8pt]
			\item Explain the major national and international scenarios/situations leading to the main wildlife forensic issues, and what the underlying forces are.
			\item Describe the major national and international laws, regulations, and policies/bodies relevant to wildlife forensics, and what roles they play.
			\item Explain at least ten case studies in wildlife forensics, including the drivers, legal issues, and implications for wildlife.
			\item Apply molecular species identification techniques for wildlife forensic cases, including an explanation of each step, how it works, and why it is used.
			\item Explain the ways in which genetic techniques can be applied to wildlife forensics, as well as the strengths and weaknesses of each approach.
			\item Give a scientific presentation that effectively teaches the class a topic, and lead the class in an effective and engaging discussion on that topic.\\
		\end{itemize}

	
	\vspace{0.3cm}
	%-----------------------------------%
	%      Marking Scheme               %
	%-----------------------------------%
	\textbf{Marking Scheme:}
		\begin{table}[H]
		\centering
			\begin{tabular}{l l}
				\toprule
				\textbf{Component} & \textbf{\% of Final Grade}\\
				\midrule
				Case study presentation \& discussion & 20\%\\
				\addlinespace
				Lab books (2 $\times$ 12.5\% each) & 25\%\\
				\addlinespace
				Midterm exam & 25\%\\
				\addlinespace
				Final Exam & 30\%\\
				\midrule
				\textbf{Total} & \textbf{100\%}\\
				\bottomrule
			\end{tabular}
		\end{table}	


	\newpage
	%----------------------------------------%
	%  Description of Course Components      %
	%----------------------------------------%
	\textbf{Description of Course Components:}
		\begin{table}[H]
			\begin{tabular}{@{} >{\raggedright}p{2.5cm} p{13.7cm}}
				\textbf{Case study presentation \& discussion:} & The class will be divided into 10 teams of 2--3 students each. Each team will then be assigned a specific case study in wildlife forensics. Some introductory readings will also be provided to help each team get started. Each team must then:
				\begin{enumerate}
					\item Conduct further research on this topic to obtain a thorough understanding of it. This will require much more research than just reading the assigned documentation.
					\item Effectively teach this case study to the rest of the class, on the assigned day. Through this, the rest of the class should obtain a more thorough understanding of the topic than they already have based on the assigned readings.
					\item Lead an active discussion with the class on this topic.
				\end{enumerate}\\
				\addlinespace
				\textbf{Lab books} & Your lab books will be collected and reviewed twice throughout the term to ensure that you are using them properly. Lab books are a key part of any type of lab work, and particularly so for a forensic technician whose lab book can be subpoenaed and used as evidence in a trial. Your lab book should be clean, clear, and easy to follow. It should explain not just WHAT you did, but also WHY you did it. Further details will be provided in class. If you miss a lab day, you should arrange with a friend to have them conduct the necessary processes on your sample---there will not be the time or opportunity to make these up. You should then write the appropriate information in your lab book but make it clear that another person (include their name) actually conducted that day's work on your sample.\\
				\addlinespace
				\textbf{Exams} & There will be two exams for this course: a midterm exam and a cumulative final exam. The midterm exam will be conducted during class time mid-way through the course, to assess understanding at this stage. The final exam will be scheduled by the Registrar during the formal exam period. Please see the University Special Exams policy (Academic Reg. \#10) for further information. These exams will generally be essay-style questions, with minimal or no use of multiple choice or fill in the blank. The rationale is that I want to assess whether or not you have really obtained new knowledge, and can put it into a correct context and interpretation, as opposed to just memorizing ``stuff''. I find this is easier to do with longer essay-style questions. You will get your exams back in class, but you cannot take them with you. If you want to review them in detail, you can do so at my office.\\
			\end{tabular}
		\end{table}		
		
		
	%---------------------------------------------------------------------%
	%  Student Responsibilities, Academic Integrity, & Code of Conduct    %
	%---------------------------------------------------------------------%
	\textbf{Student Responsibilities, Academic Integrity, \& Code of Conduct}
		\begin{enumerate}[topsep=-8pt]
			\item Treat your colleagues and instructor with respect and give others your attention when they are speaking.
			\item Smart phones may be used to take notes, but all the sounds must be turned off and you should not receive/send calls/texts, or check email during class.
			\item \textbf{Academic integrity: As in all courses, plagiarism and cheating will not be tolerated.} You must hand in your own work, written in your own words. Plagiarism will be dealt with according to policies outlined in the Academic Calendar. It is your responsibility to familiarize yourself with Saint Mary's policies on Academic Integrity by consulting the ``\emph{Academic Integrity and Student Responsibility}'' (p. 14--22) and ``\emph{Academic Regulations}'' (p. 23--38) sections of the \href{https://smu.ca/webfiles/AcademicCalendar2019-2020Undergraduate(PDF).pdf}{\textcolor{blue}{Academic Calendar}}, in order to be well informed on the consequences of dishonest behaviour.  
			\item Technology in the classroom: Please do not record lectures without my direct approval.
			\item \textbf{Late policy:} You will be penalized 5\% \emph{per day} that an assignment is late. Weekend days are included in this (\emph{i.e.,} Saturday and Sunday each count as one day).
		\end{enumerate}
		
			
\newpage		
%------------------------------------%
% Course Content & Schedule          %
%------------------------------------%
\textbf{Course Content and Planned/Tentative Schedule:}\\
\emph{The schedule below is TENTATIVE. We will try to stick to this schedule, but it is likely that we will get off-track at one point or another. Necessary changes to the schedule will be made accordingly, and you will be notified of any changes during class hours.}	

	\begin{table}[H]
		\footnotesize
		\centering
		\rowcolors{2}{white}{black!15}
		\begin{tabular}{l p{6cm}}
			\toprule
			\textbf{Day} & \textbf{Topic}\\
			\midrule
			1. Wednesday, Sep. 4	& \makecell[tl]{Lecture: Introduction to course\\ Lab: None}\\
			\addlinespace
			2. Monday, Sep. 9		& Overview of major international issues\\
			\addlinespace
			3. Wednesday, Sep. 11	& \makecell[tl]{Lecture: Case study \#1\\ Lab: Sample collection}\\
			\addlinespace
			4. Monday, Sep. 16 		& International law and regulations\\
			\addlinespace
			5. Wednesday, Sep. 18 	& \makecell[tl]{Lecture: Case study \#2\\ Lab: DNA extraction}\\
			\addlinespace
			6. Monday, Sep. 23 		& NO CLASS, I'm away\\
			\addlinespace
			7. Wednesday, Sep. 25 	& \makecell[tl]{Lecture: Case study \#3\\ Lab: DNA quantity \& quality}\\
			\addlinespace
			8. Monday, Sep. 30 		& Major North American issues\\	
			\addlinespace
			9. Wednesday, Oct. 2	& \makecell[tl]{Lecture: Case study \#4\\ Lab: PCR}\\
			\addlinespace
			10. Monday, Oct. 7 		& North American laws and regulations\\
			\addlinespace
			11. Wednesday, Oct. 9	& \makecell[tl]{Lecture: Case study \#5\\ Lab: PCR check \& clean-up}\\
			\addlinespace
			12. Monday, Oct. 14 	& NO CLASSES, Thanksgiving\\
			\addlinespace
			13. Wednesday, Oct. 16 	& \makecell[tl]{Lecture: Case study \#6\\ Lab: Sequencing PCR}\\
			\addlinespace
			14. Monday, Oct. 21 	& Molecular species ID\\
			\addlinespace
			15. Wednesday, Oct. 23 	& \makecell[tl]{Lecture: \textbf{Midterm exam}\\ Lab: None}\\
			\addlinespace
			16. Monday, Oct. 28 	& Population assignment\\
			\addlinespace
			17. Wednesday, Oct. 30 	& \makecell[tl]{Lecture: Case study \#7\\ Lab: Sequencing clean-up \& CE}\\
			\addlinespace
			18. Monday, Nov. 4 		& Individual Assignment\\
			\addlinespace
			19. Wednesday, Nov. 6 	& \makecell[tl]{Lecture: Case study \#8\\ Lab: Sequence editing and BLAST search}\\
			\addlinespace
			20. Monday, Nov. 11 	& NO CLASSES, Fall Break\\
			\addlinespace
			21. Wednesday, Nov. 13 	& NO CLASSES, Fall Break\\
			\addlinespace
			22. Monday, Nov. 18 	& Illegal wildlife trade online\\
			\addlinespace
			23. Wednesday, Nov. 20 	& \makecell[tl]{Lecture: Case study \#9\\ Lab: Phylogenetic analysis}\\
			\addlinespace
			24. Monday, Nov. 25 	& How to change\\
			\addlinespace
			25. Wednesday, Nov. 27 	& \makecell[tl]{Lecture: Case study \#10\\ Lab: }\\
			\addlinespace
			26. Monday, Dec. 2 		& Animal rights and legislation\\
			\addlinespace
			27. Wednesday, Dec. 4 	& \makecell[tl]{Lecture: Review\\ Lab: }\\
			\bottomrule
		\end{tabular}
	\end{table}	
 

	\newpage
	%-------------------------%
	%  Missed Classes         %
	%-------------------------%
	\textbf{Missed Classes:}\\
	SMU faculty no longer accept ``sick notes'' for missed days of class or exams. Instead, if students miss a day of class, particularly when there was something due that day (e.g., a research presentation or mid-term exam), they need to read, print out, fill out, and sign a copy of the \href{http://www.smu.ca/webfiles/Declaration_of_Extenuating_Circumstances_withinTerm.pdf}{\textcolor{blue}{Declaration of Extenuating Circumstances}}. This should then be submitted to the professor, and they will keep a copy, and also give a copy to the Science Advising Centre for your records. 
	
	Students who miss the mid-term exam need to follow the instructions above, and then make an appointment to re-take the exam. There are pre-set times for this, which can be found by searching for ?missed exam dates? on the SMU website. Arrange this with me (the professor), and I will ensure that the correct materials are in the right place at the right time. If a student misses the final exam, then the university follows Academic Regulation \#10 from the \href{https://smu.ca/webfiles/AcademicCalendar2019-2020Undergraduate(PDF).pdf}{\textcolor{blue}{Academic Calendar}} (p. 30). For this, students do not interact with the professor. Instead, they consult with the Science Advising Centre, who then contacts the professor to develop a solution.


	\vspace{0.3cm}
	
	%----------------------------------------------%
	%  Accessibility                               %
	%----------------------------------------------%
	\textbf{Accessibility:}\\
	The Fred Smithers Centre establishes individualized support services to help students with physical, medical, and learning disabilities. Accommodations work best for all concerned if the student comes forward to the Smithers Centre early.  Students are encouraged to seek more information by visiting the Centre, or its \href{http://www.smu.ca/campus-life/fred-smithers-centre.html}{\textcolor{blue}{website}}.	 
\end{document}